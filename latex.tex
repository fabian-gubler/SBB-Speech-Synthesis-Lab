



% Options for packages loaded elsewhere
\PassOptionsToPackage{unicode}{hyperref}
\PassOptionsToPackage{hyphens}{url}
%
\documentclass[
]{article}
\usepackage{amsmath,amssymb}
\usepackage{lmodern}
\usepackage{iftex}
\ifPDFTeX
  \usepackage[T1]{fontenc}
  \usepackage[utf8]{inputenc}
  \usepackage{textcomp} % provide euro and other symbols
\else % if luatex or xetex
  \usepackage{unicode-math}
  \defaultfontfeatures{Scale=MatchLowercase}
  \defaultfontfeatures[\rmfamily]{Ligatures=TeX,Scale=1}
\fi
% Use upquote if available, for straight quotes in verbatim environments
\IfFileExists{upquote.sty}{\usepackage{upquote}}{}
\IfFileExists{microtype.sty}{% use microtype if available
  \usepackage[]{microtype}
  \UseMicrotypeSet[protrusion]{basicmath} % disable protrusion for tt fonts
}{}
\makeatletter
\@ifundefined{KOMAClassName}{% if non-KOMA class
  \IfFileExists{parskip.sty}{%
    \usepackage{parskip}
  }{% else
    \setlength{\parindent}{0pt}
    \setlength{\parskip}{6pt plus 2pt minus 1pt}}
}{% if KOMA class
  \KOMAoptions{parskip=half}}
\makeatother
\usepackage{xcolor}
\IfFileExists{xurl.sty}{\usepackage{xurl}}{} % add URL line breaks if available
\IfFileExists{bookmark.sty}{\usepackage{bookmark}}{\usepackage{hyperref}}
\hypersetup{
  hidelinks,
  pdfcreator={LaTeX via pandoc}}
\urlstyle{same} % disable monospaced font for URLs
\setlength{\emergencystretch}{3em} % prevent overfull lines
\providecommand{\tightlist}{%
  \setlength{\itemsep}{0pt}\setlength{\parskip}{0pt}}
\setcounter{secnumdepth}{-\maxdimen} % remove section numbering
\ifLuaTeX
  \usepackage{selnolig}  % disable illegal ligatures
\fi

\author{}
\date{}

\begin{document}

\hypertarget{results-todo}{%
\section{Results (TODO)}\label{results-todo}}

\hypertarget{figures-to-include}{%
\subsection{Figures to include}\label{figures-to-include}}

\begin{itemize}
\tightlist
\item
  Table 1: A table comparing the Word Error Rate (WER) of the model at
  different stages: baseline (0 shot), training with human-recorded
  samples only, and various levels of synthetic data inclusion.
\item
  Figure 7: A bar or line graph showing the change in performance (WER)
  as the amount of synthetic data used in training is increased. This
  would provide a visual representation of the trend.
\item
  Figure 9: A confusion matrix to visually show where the model tends to
  make errors, in terms of predicting certain words incorrectly.
\item
  Figure 10: An example of the model's prediction, with the original
  text, the predicted text, and highlighting errors, could serve as a
  case study to underline the strengths and weaknesses.
\end{itemize}

\hypertarget{added-evaluation}{%
\subsection{Added Evaluation:}\label{added-evaluation}}

Quantiative Analysis:

\begin{itemize}
\tightlist
\item
  Confusion Matrix: This analysis could be useful to visualize the
  model's performance and understand which words are often confused with
  each other.
\end{itemize}

Qualitative Analysis:

\begin{itemize}
\item
  Manual Error Analysis: You could manually review a subset of the
  transcriptions to understand the nature of the errors made by the
  model. This could be insightful, especially when the model makes
  mistakes that are not captured by the metrics.
\item
  Case Studies: You could present specific examples where the model
  performed exceptionally well or poorly. This could provide more
  context to the quantitative results and make the findings more
  tangible. This includes looking at specific foreign accents.
\end{itemize}

\hypertarget{results}{%
\section{Results}\label{results}}

The results section serves to present the key findings of the project
and primarily focuses on the performance comparison of the Conformer-CTC
model under various training conditions. It delineates the model's
performance when trained with different types and amounts of synthetic
data, and how this compares to both the baseline model and the model
trained only with human-recorded samples.

\hypertarget{model-performance-with-data-augmentation}{%
\subsection{Model Performance with Data
Augmentation}\label{model-performance-with-data-augmentation}}

This subsection provides a comprehensive analysis of how the inclusion
of synthetic data in training impacted the model's performance. It
begins by detailing the performance of the Conformer-CTC model at
baseline (0-shot) and then explores the model's performance as synthetic
data is incrementally added to the training set.

A table will be included to display the performance metrics at each
step. This table will show the performance metrics for each increment of
synthetic data added, allowing for easy comparison between the various
levels of synthetic data inclusion.

\hypertarget{comparison-with-human-recorded-samples}{%
\subsection{Comparison with Human-Recorded
Samples}\label{comparison-with-human-recorded-samples}}

In this subsection, the performance of the model trained with synthetic
data is compared against the model trained only with human-recorded
samples. The discussion should outline how the use of synthetic data in
training contributed to the improvement in model performance.

A comparative table can be included here to clearly show the difference
in performance metrics between the models trained with and without
synthetic data.

\hypertarget{accent-specific-performance}{%
\subsection{Accent-Specific
Performance}\label{accent-specific-performance}}

This part of the section delves into the impact of different accents on
model performance. It discusses the model's proficiency in recognizing
speech with diverse accents, focusing particularly on German accents and
how the model performed when trained with German-accented synthetic
samples.

A table detailing the performance of the model on various accents should
be included here. This table can provide the performance metrics for
each type of accent, highlighting the model's strengths and potential
areas for improvement in accent recognition.

\hypertarget{contributions-of-synthetic-data}{%
\subsection{Contributions of Synthetic
Data}\label{contributions-of-synthetic-data}}

This subsection aims to pinpoint the specific aspects of the synthetic
dataset that contributed most to the improvement in model performance.
It analyzes the performance data to determine whether certain features
of the synthetic data---such as particular accents or commandos---were
especially beneficial.

\hypertarget{qualitative-analysis}{%
\subsection{Qualitative Analysis}\label{qualitative-analysis}}

This subsection provides specific examples of the model's predictions to
highlight its strengths and weaknesses. This can include instances where
the model excelled or struggled in recognizing particular commandos or
accents, providing a more qualitative perspective to complement the
performance metrics.

By structuring the results section in this manner, it would be easier
for you to plug in the specific results once they are available. You
could then tailor the narrative in each subsection based on the actual
findings.

\hypertarget{discussion}{%
\section{Discussion}\label{discussion}}

\hypertarget{recapitulation-of-key-findings}{%
\subsection{Recapitulation of Key
Findings}\label{recapitulation-of-key-findings}}

Start this section by succinctly summarizing the primary outcomes of
your experiments and results. Highlight the most significant findings,
improvements over the baseline, the impacts of data augmentation, and
the model's performance on different accents.

For example: ``Our work successfully demonstrated that the application
of synthetic data augmentation can significantly enhance the performance
of a Conformer-CTC model in automatic speech recognition tasks.
Specifically, we observed a significant reduction in Word Error Rate
compared to the baseline, particularly in instances where we
incorporated German-accented synthetic samples\ldots{}''

\hypertarget{interpretation-and-implications}{%
\subsection{Interpretation and
Implications}\label{interpretation-and-implications}}

In this section, provide your interpretations of the results and discuss
the implications. Evaluate the real-world impact and potential
applications of your findings.

For instance: ``The notable improvements in our model's performance,
particularly when trained with synthetic data augmented with diverse
accents, signal a significant breakthrough in overcoming the challenges
posed by multi-accented and diverse speech. This finding has profound
implications for industries and applications where\ldots{}''

\hypertarget{comparison-with-existing-literature}{%
\subsection{Comparison with Existing
Literature}\label{comparison-with-existing-literature}}

Briefly compare your findings with existing literature. Discuss how your
work aligns with or diverges from previous studies.

For example: ``While our results align with prior research indicating
the effectiveness of data augmentation in speech recognition tasks, our
work further extends this understanding by demonstrating that\ldots{}''

\hypertarget{strengths-and-limitations}{%
\subsection{Strengths and Limitations}\label{strengths-and-limitations}}

Acknowledge the strengths and limitations of your work. This could
include aspects of the methodology, the model used, or the
generalizability of your findings.

For instance: ``While our research effectively highlights the potential
of synthetic data augmentation in improving speech recognition
performance, it is not without limitations. One potential limitation
is\ldots{}''

\hypertarget{recommendations-for-future-research}{%
\subsection{Recommendations for Future
Research}\label{recommendations-for-future-research}}

Outline the opportunities for further research, based on your findings
and experiences.

For instance: ``Given the promising results achieved through the
incorporation of synthetic data, future research could explore the use
of more sophisticated text-to-speech systems to generate even more
realistic synthetic speech. Additionally, the exploration of other
models or hybrid models could yield\ldots{}''

\hypertarget{conclusion}{%
\subsection{Conclusion}\label{conclusion}}

Conclude the discussion section by summarizing the main points,
reaffirming the significance of your findings, and providing a clear
take-home message for the reader.

For instance: ``In conclusion, our research illustrates the
transformative potential of synthetic data augmentation in advancing
automatic speech recognition tasks. Despite the challenges of diverse
accents and limited data, we demonstrated that it's possible to achieve
significant improvements using an innovative data augmentation
strategy\ldots{}''

\end{document}
